\documentclass{article}
\usepackage{booktabs}
\usepackage{graphicx}
\usepackage{multicol} % Enable multiple columns.
\usepackage{float}
\usepackage[margin=2cm]{geometry} % Set the margins to 2 cm.

% Begin the document.
\begin{document}

% Create the title and remove the 2 cm of whitespace above it.
\title{\vspace{-2.0cm}Concurrent Report}

% Create the authors.
\author{Ainsley Rutterford \\ \texttt{ar16478@my.bristol.ac.uk} \\ Computer Science
    \and Harry Waugh \\ \texttt{hw16470@my.bristol.ac.uk} \\ Computer Science}

% Draw the title, authors, and date.
\maketitle

% Make sure that everything from this point onwards is two columns.
\begin{multicols}{2}

% Create a new section.
\section{Functionality and Design}
Our system currently uses up to 8 workers to evolve the Game-of-Life repeatedly. Our system is 
deadlock-free, implements the correct button, board orientation, and LED behaviour, and can process 
images up to 1264x1264 pixels using memory on both tiles.

\vspace{5mm}

The biggest problem we encountered was trying to run images bigger than 512x512. Originally we were 
reading the bits from the .pgm file into a 2 dimensional array of unsigned chars, before 'packing' 
this array into an array of unsigned chars that represented each pixel by a bit instead of a byte. 
This packing process allowed us to save space as we only had to store an array which was 1/8th the 
size of the original array. We proccessed the Game-of-Life on this array. Eventually we had to 
free even more space in order to run larger images. To do this, we changed the dataInStream function, 
so that it read the pixel values straight into the smaller array, representing each pixel as a 
single bit. At this point we could process images up to 688x688. We then then realised that in
\texttt{dataInStream}, we didn't have to store an unsigned char array at all. We simply processed 
each byte and sent each byte to the distributer as we read from the file. This meant that instead 
of storing an entire array of unsigned chars of size [ImageHeight][ImageWidth / 8], we stored a single 
unsigned char.

\vspace{5mm}

We also encountered problems while implementing a timer. The first problem was that the timer seemed 
to overflow roughly every 42 seconds. We decided to create a function that would compare two times 
given, and would return whether or not the timer has overflowed. We would then call this function when 
the board was tilted, and if the function returned true, we would increment a counter which counted 
how many times the timer has overflowed. The time displayed would be the timers current value added 
to the overflow value multiplied by the counter value. Using this method, our clock no longer 
overflowed.

\vspace{5mm}

Early on in the development of our system, our workers functioned differently. Each worker would work
on a single byte of data. We would send each worker a 3x3 array of unsigned chars, with the middle 
char being the char that the worker would work on. The worker would then send only the completed
char back. We realised that this was very inefficient as each worker required 8 extra bytes to work
on a single byte. Image processing was very slow using this method so we decided to update
our workers to work on a whole strip of an image at once. This meant that for whole chunk of an image
to be worked on, only an extra strip of pixels at the top and bottom of the strip had to be sent as
the sides of the image wrap round to each other. Using this new worker strategy, our image processing
was up to 11 times faster.

% Create a new section.
\section{Tests and Experiments}
For each image provided to us, we will show the initial image along side the image after 1, 2 and 100
iterations respectively. We will also include randomly generated 1024x1024 and 1264x1264 images. The
results are shown in Figure...

\vspace{5mm}

We carried out experiments comparing the time taken for an image to be processed when using different
workers. We experimented with using 1, 2, 4, and 8 workers. The results are show in Figure... 

\vspace{5mm}

One limitation of our system is that it can only work on images that have a width that is divisible by
8. This is due to the fact that we 'pack' 8 pixels into an unsigned char to represent each pixel by
a single bit. Another limitation of our system is the size of images it can process, due to the limited
amount of memory provided by the board. In order to process images any larger than 1272x1272, we would
have to either, attach another board to our own, or remove the functionality of the buttons, LEDs and
orientation sensors of the board.

\end{multicols}

\newpage

\newgeometry{margin=1cm}

\begin{multicols}{2}

\begin{figure}[H]
\centering
\includegraphics[width=3cm]{16initial.png}
\includegraphics[width=3cm]{64onenew.png}
\includegraphics[width=3cm]{16initial.png}
\caption{16x16 images.}
\end{figure}

\begin{figure}[H]
\centering
\includegraphics[width=3cm]{16initial.png}
\includegraphics[width=3cm]{64onenew.png}
\includegraphics[width=3cm]{16initial.png}
\caption{128x128 images.}
\end{figure}

\begin{figure}[H]
\centering
\includegraphics[width=3cm]{16initial.png}
\includegraphics[width=3cm]{64onenew.png}
\includegraphics[width=3cm]{16initial.png}
\caption{256x256 images.}
\end{figure}

\begin{figure}[H]
\centering
\includegraphics[width=3cm]{16initial.png}
\includegraphics[width=3cm]{64onenew.png}
\includegraphics[width=3cm]{16initial.png}
\caption{512x512 images.}
\end{figure}

\begin{figure}[H]
\centering
\includegraphics[width=3cm]{16initial.png}
\includegraphics[width=3cm]{64onenew.png}
\includegraphics[width=3cm]{16initial.png}
\caption{1024x1024 images.}
\end{figure}

\begin{figure}[H]
\centering
\includegraphics[width=3cm]{16initial.png}
\includegraphics[width=3cm]{64onenew.png}
\includegraphics[width=3cm]{16initial.png}
\caption{1264x1264 images.}
\end{figure}

\end{multicols}

\begin{figure}[H]
\centering
\includegraphics[width=3cm]{16initial.png}
\includegraphics[width=3cm]{16initial.png}
\includegraphics[width=3cm]{16initial.png}
\includegraphics[width=3cm]{64onenew.png}
\includegraphics[width=3cm]{64onenew.png}
\includegraphics[width=3cm]{64onenew.png}
\caption{64x64 images.}
\end{figure}


\newgeometry{margin=1cm}

% Create a new section.
\section{How to create a bulleted list}
This is the Second Section. Let's also see if this text wraps. Will this be below or beside Introduction? Here is how to create a list:

% Create a list.
\begin{itemize}
\item First bullet point
\item This is the second bullet point.
\item And so on...
\end{itemize}

Then we can carry on the section. But we must be careful because the beginning of this line will be indented as it is after a list.

% End the column and start a new one.
%\columnbreak

% Create a new section.
\section{New Section}
This is a new Section. This one should wrap too. This is the third section.

% End the two column formatting.
%\end{multicols}

% End the document.
\end{document}

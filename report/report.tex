\documentclass{article}
\usepackage{booktabs}
\usepackage{graphicx}
\usepackage{multicol} % Enable multiple columns.
\usepackage[margin=2cm]{geometry} % Set the margins to 2 cm.

% Begin the document.
\begin{document}

% Create the title and remove the 2 cm of whitespace above it.
\title{\vspace{-2.0cm}Concurrent Report}

% Create the authors.
\author{Ainsley Rutterford \\ \texttt{ar16478@my.bristol.ac.uk} \\ Computer Science
    \and Harry Waugh \\ \texttt{hw16470@my.bristol.ac.uk} \\ Computer Science}

% Draw the title, authors, and date.
\maketitle

% Make sure that everything from this point onwards is two columns.
\begin{multicols}{2}

% Create a new section.
\section{Functionality and Design}
Our system currently uses up to 8 workers to evolve the Game-of-Life repeatedly. Our system is 
deadlock-free, implements the correct button, board orientation, and LED behaviour, and can process 
images above 512x512 pixels using memory on both tiles.

The biggest problem we encountered was trying to run images bigger than 512x512. Originally we were 
reading the bits from the .pgm file into a 2 dimensional array of unsigned chars, before 'packing' 
this array into an array of unsigned chars that represented each pixel by a bit instead of a byte. 
This packing process allowed us to save space as we only had to store an array which was 1/8th the 
size of the original array. We used this array to process the Game-of-Life on. Eventually we had to 
free even more space in order to run larger images. To do this, we changed the dataInStream function, 
so that it read the pixel values straight into the smaller array, representing each pixel as a 
single bit. 

We also encountered problems while implementing a timer. The first problem was that the timer seemed 
to overflow roughly every 42 seconds. We decided to create a function that would compare two times 
given, and would return whether or not the timer has overflowed. We would call this function when 
the board was tilted, and if the function returned true, we would increment a counter which counted 
how many times the timer has overflowed. The time displayed would be the timers current value added 
to the overflow value multiplied by the counter value. Using this method, out clock no longer 
overflowed.

% Create a new section.
\section{How to create a bulleted list}
This is the Second Section. Let's also see if this text wraps. Will this be below or beside Introduction? Here is how to create a list:

% Create a list.
\begin{itemize}
\item First bullet point
\item This is the second bullet point.
\item And so on...
\end{itemize}

Then we can carry on the section. But we must be careful because the beginning of this line will be indented as it is after a list.

% End the column and start a new one.
\columnbreak

% Create a new section.
\section{New Section}
This is a new Section. This one should wrap too. This is the third section.

% End the two column formatting.
\end{multicols}

% End the document.
\end{document}
